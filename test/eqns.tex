\documentclass{article}
\begin{document}

Needs a number
\begin{equation}\label{ab3}
\,\varphi\,=\,\left|\,
\begin{array}{c}
\psi_1 \\
\psi_2
\end{array}\,\right|\,; \quad {\rm and} \quad
\,\chi\,=\,\left|\,
\begin{array}{c}
\psi_3 \\
\psi_4
\end{array}\,\right|\,; \quad {\rm or} \quad
\,\eta\,=\,\left|\,
\begin{array}{c}
\tilde{\psi}_1 \\
\tilde{\psi}_2
\end{array}\,\right|\,; \quad {\rm and} \quad
\,\lambda\,=\,\left|\,
\begin{array}{c}
\tilde{\psi}_3 \\
\tilde{\psi}_4
\end{array}\,\right|\,;
\end{equation}
more text

\section{Inline equations}

First begin with simple \verb#$# delimited equation such as
$x+y=w$ as an example.  All the equations in this section should
look identical.

Next how about a simple \verb#\begin{math}# delimited equation such as
\begin{math}x+y=w\end{math} as an example.  All the equations in this section should
look identical.

Now consider \verb#\(# and \verb#\)# delimited equation such as
\(x+y=w\) as an example.  All the equations in this section should
look identical.

\section{Unnumbered equations}

I will start with a simple \verb#$$# wave equation
that will have no number
$$
\nabla^2 \phi -{1\over c} {\partial \phi\over \partial t}  = 0
$$
This is followed by a \verb#displaymath# environment
\begin{displaymath}
\nabla^2 \phi -{1\over c} {\partial \phi\over \partial t}  = 0
\end{displaymath}
Here is an example of the \verb#\[# displaymath environment
\[
\nabla^2 \phi -{1\over c} {\partial \phi\over \partial t}  = 0
\]


\section{Numbered equations}
Next comes an \verb#equation# environment
\begin{equation}
\nabla^2 \phi -{1\over c} {\partial \phi\over \partial t}  = 0
\end{equation}
Note that \verb#\nonumber# in an \verb#equation# environment still gets
an equation number
\begin{equation}
\nabla^2 \phi -{1\over c} {\partial \phi\over \partial t}  = 0 \nonumber
\end{equation}

\section{Testing equation array}

Here the equation array is being tested.  This equation has no equation number
and is about as simple as an equation array can get.
\begin{eqnarray*}
z & = & w + x + \\
  &   & 5w - 8c 
\end{eqnarray*}

Here the equation array is being tested.  This equation has equation numbers
and is almost as simple as an equation array can get.
\begin{eqnarray}
z & = & w + x + \\
  &   & 5w - 8c 
\end{eqnarray}

Here the equation array is being tested.  This equation the first and third equations numbered
\begin{eqnarray}
z & = & w + x  \\
z & = & w + x \nonumber\\
z & = & w + x  
\end{eqnarray}

Here the equation array is being tested.  This checks for a bug when 
\verb#\nonumber# is present in an \verb#\begin{eqnarray*}# environment.  
No equations should be numbered.
\begin{eqnarray*}
z & = & w + x  \\
z & = & w + x \nonumber\\
z & = & w + x  
\end{eqnarray*}

\section{Testing math environment closing}

For a while, getting \texttt{latex2rtf} to contain all the math elements
to the enclosing was a major headache.  It is working for now.  Here are
a few test cases.

Case 1 \textit{s$_c+1$} or {\it s$_c$}

Case 2 $a+\bf R$ roman \verb#$# type

Case 3 \(a+\bf R\) roman type

Case 4 \[a+\bf R\] roman type

Case 5 \begin{math}a+\bf R\end{math} roman type

Case 6 \begin{equation}a+ \bf R\end{equation} roman type

Case 7 \begin{eqnarray}a+ \bf R\end{eqnarray} roman type

Case 8 $$a+ \bf R$$ roman type

\end{document}
