%%%%%%%%%%%%%%%%%%%%%%%%%                -*-LaTeX-*-
%%%   Moroz_OT2.tex   %%%
%%%%%%%%%%%%%%%%%%%%%%%%%
%%
%%  Date:   9-Aug-2002
%%  Author: wd (Wolfgang.Dobler@ncl.ac.uk)
%

\documentclass[a4paper,12pt]{article}

\usepackage[T2A,OT1]{fontenc}
\usepackage[ot2enc]{inputenc}
\usepackage[russian,english]{babel}

\sloppy                         % Hyphenation is a problem..

\author{A. S. Pushkin}
\title{{\cyr Zimnee utro} (winter morning)}

\begin{document}
\maketitle

\section{A section}

\noindent{\cyr 
Moroz i solnce, den\cyrsftsn{} chudesny\U{i}!\\
Eshch\"e ty dremlesh\cyrsftsn{}, drug prelestny\U{i}?\\
Pora, krasavica, prosnis\cyrsftsn!}

\subsection{Accented characters}
You can still use accents and diacritica as usual:
\begin{quote}
  \"Uberm\"a{\ss}iger Lik\"orgenuss in der \"Offentlichkeit f\"uhrt
  zu \"Arger 
\end{quote}
\begin{quote}
  \^Etre ou ne pas \^etre --- c'est une question tr\`es int\'eressante
  en Fran\c{c}ais
\end{quote}

\subsection{\cyr Shrifty}
\cyr {\sc Konechno} {\bf vse} {\em varianty} {\TeX}ovskih {\sl shriftov}
dostupny i s {\sf kirillice\U{i}}.

\end{document}

