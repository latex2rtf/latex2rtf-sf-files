%%%%%%%%%%%%%%%%%%%%%%%%%%%%    -*- mode: LaTeX; coding: utf-8; -*-
%%%  Moroz_koi koi.tex   %%%
%%%%%%%%%%%%%%%%%%%%%%%%%%%%
%%
%%  Date:   9-Aug-2002
%%  Author: wd (Wolfgang.Dobler@ncl.ac.uk)
%

\documentclass[a4paper,12pt]{article}

\usepackage[T2A]{fontenc}
\usepackage[utf8]{inputenc}   % older versions of ucs package
%\usepackage[utf8x]{inputenc}  % more recent versions (at least>=2004-17-10)
\usepackage[russian]{babel}

\sloppy                         % Hyphenation is a problem..

\author{A. S. Pushkin}
\title{Зимнее утро (winter morning)}

\begin{document}
\maketitle

\section{A section}

\noindent
Мороз и солнце, день чудесный!\\
Ещё ты дремлешь, друг прелестный?\\
Пора, красавица, проснись!

\subsection{Accented characters}
You can still use accents and diacritica as usual:
\begin{quote}
  Übermäßiger Likörgenuss in der Öffentlichkeit führt
  zu Ärger 
\end{quote}
\begin{quote}
  Être ou ne pas être --- c'est une question très intéressante
  en Français
\end{quote}

\subsection{Шрифты}
{\sc Конечно} {\bf все} {\em варианты} {\TeX}овских {\sl шрифтов}
доступны и с {\sf кириллицей}.

\end{document}

%% Emacs-Mule quick reference:
%% C-\           toggles input methods (cyrillic/ascii)
%% C-x RET C-\   switches to (yet) another input method
%% C-x RET f     changes the buffer-coding system


% Make Emacs switch to russian keyboard layout
%
%% [Better not to use this 1] Local Variables:
%% [Better not to use this 2] eval: (set-input-method 'cyrillic-jcuken)
%% [Better not to use this 3] End:
